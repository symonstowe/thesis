% FILE:    abstract.tex
% AUTHOR:  Symon Stowe 
% DATE:    2020-01-21
%--------------------------------------------------------

% page number at bottom
\thispagestyle{plain}

%\section*{Summary}
\begin{abstract}
Electrical impedance tomography (EIT) is a medical imaging technology 
that uses boundary electrodes to inject stimulus currents and measure 
the resulting potential distributions. These potentials are measured using 
electrodes which are in turn used to reconstruct 
conductivity changes within the body. EIT has been studied for its ability 
to image both the flow of blood and the delivery of blood to a 
tissue (perfusion). However, cardiac-related signals are challenging to 
image accurately due to their small amplitude and the limited sensitivity 
of EIT systems to impedance changes deep in the body. This thesis presents techniques to improve 
perfusion imaging with EIT by generating more accurate meshes and using 
internal electrodes to obtain higher sensitivity in the centre of the body. 
This work develops 
a tool to generate accurate, customized meshes from diagnostic 
computed tomography (CT) images, and a technique to reconstruct images 
with internal electrodes. Custom models 
reconstructed the location of impedance changes due to ventilation with 
higher accuracy, and internal electrodes yielded an increase 
in internal sensitivity over traditional external configurations. 
Shifting the position of electrodes on an internal probe by as little as 1\% of the tank radius
in simulation created artefacts in images reconstructed 
using existing approaches without motion correction. 
A novel technique to correct for probe motion is presented 
that improved reconstruction accuracy and reduced background
noise compared to existing techniques. %in reconstructed images with probe movement of up to 10\% of 
%the model radius.% compared to existing techniques. 
The presented work contributes to increasing internal sensitivity 
of EIT measurements and demonstrates that refined meshes and internal electrodes 
may improve measures of perfusion and help to make EIT a viable tool 
for continuous perfusion monitoring at the bedside. 

\end{abstract}

%\vspace{4mm}
