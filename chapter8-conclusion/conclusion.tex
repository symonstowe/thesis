\chapter{Conclusion}

This thesis investigated mesh refinement and internal electrodes 
as tools to advance perfusion imaging. 
EIT has very low sensitivity in the centre of the thorax where 
majority of the perfusion changes of interest occur. 
The source of the cardiosynchronous EIT signal is also 
poorly understood, and many factors may contribute to 
the measured perfusion signal. 
This thesis presents several contributions to the field of 
perfusion imaging in EIT.
We investigated the limitations of current EIT perfusion imaging
and compared the difference between measures of perfusion using 
a contrast agent and those relying on the cardiosynchronous signal.
A technique to control mesh refinement was developed to create meshes
with adequate sensitivity accuracy throughout an entire model. This 
technique can also guide refinement strategies in high sensitivity regions  
surrounding internal electrodes. 
A 3D EIT configuration is presented with four internal electrodes that gives high
sensitivity in the centre of the model, and accurately reconstructs a conductive
target in the presence of Gaussian measurement noise.
Finally, we developped a method to calculate movement of 
an internal probe and correct for positioning error in a model. Electrode 
motion can add artefacts
to images that overpower the 
component of interest, and background noise to the reconstructed images. 
This technique uses charactersitic information in the reconstructed 
image related to probe movement to estimate the actual probe location 
and create a corrected model. The corrected model 
reduced the effect of movement on reconstructed images. 

\section{Summary of Findings}

\Fref{chap:chapter-3} compared several EIT measures of perfusion 
to investigae the potential of EIT as a device to monitor 
perfusion. 
Despite the different signal origins, both filtering-, and bolus-based 
techniques 
resulted in similar perfusion estimates. We found that perfusion 
estimation with EIT is variable, and hypothesize that 
improved sensitivity and reconstruction accuracy could improve perfusion 
measures and clinical viability. 

%One of the challenges with  both bolus- and filtering-based perfusion 
%imaging is the unwanted contribution of blood flow in the heart. 
%In 2D images that have low sensitivity in the centre of a subject,
%it is challenging to identify and correctly remove the contribution 
%of the heart from the perfusion signal. 

In \fref{chap:chapter-4} we explore techniques to
improve sensitivity accuracy in meshes, and ensure mesh density is adequate 
around electrodes. 
When mesh density dissipates evenly away from electrodes, we recommend 
the balance point 
of the nodes in a mesh should be towards the electrodes. 
A balance point is 
at 80\% of the distance between the lowest sensitivity region of a 
model and the electrodes minimized error when calculating sensitivity. 
This recommendation can be used to generate 
meshes that are highly accurate when calculating the sensitivity, with fewer elements
than meshes that are uniformly refined.  

In \fref{chap:chapter-5} a tool to generate accurate, custom meshes from 
CT images was presented. This tool automatically segmented the lung and external 
boundaries from CT images, and allowed manual verification of the segmented boundaries. 
The segmented boundaries were used to generate custom EIT meshes. 
Ventilation images on a small number of subjects 
showed a small improvement over generic models when 
measuring the centre of mass of the ventilated region. The presented tool 
created meshes with individualized, accurate boundaries 
for patients with CT data.

\Fref{chap:chapter-6} presented a sensitivity analysis and simulated 
reconstruction accuracy of novel electrode configurations with internal electrodes. 
These simulations show an increased sensitivity in internal regions. Reconstructions 
of a conductive object also show that internal electrodes can reconstruct 
the location of a conductive target in the presence of noise. From this work 
we determined that internal electrodes could be used in 3D to image conductivity changes 
within the body and could improve sensitivity to internal conductivity changes. 

In \fref{chap:chapter-7} we analyze the effect of motion on reconstructed images with 
internal electrodes. Moving the probe only 1\% of the model radius added artefacts
to images reconstructed without motion correction strategies. 
Available motion correction techniques reduced the impact of 
electrode movement up to 5\% of the tank boundary, but still showed an increase in 
image noise. 
We present a technique to reconstruct motion artefacts and use these images to 
estimate the true probe location. A new model with the new probe location 
was used to reconstruct images. Images reconstructed with the new technique
accurately located a conductive object when the probe moved up to 
10\% of the model radius between measurements, 
and reduced noise in all probe movement scenarios. 

\section{Future Work}

There are several avenues that we recommend be explored further from the work and projects
presented in this thesis. 

\begin{itemize}
	\item Work is currently underway to test the automatic segmentation tool on a wider
	number of subjects. We are currently working with the Peking Union Medical College in 
	Beijing, China to obtain more CT and EIT data from ARDS patients.

	\item The effect of electrode placement errors on reconstruction accuracy
	when creating custom meshes from CT images is unclear. 
	Incorrectly modelled electrodes degrade reconstruction accuracy \parencite{boyle_impact_2011},
	and incorporating the correct electrode location into the custom meshes could yield 
	more accurate EIT images. We discuss some possibilities in \fref{chap:chapter-5} to identify 
	the correct electrode locations, including using photographs to identify electrode locations
	and estimating electrode placement from the movement 
	jacobian. Chest expansion can have a large effect on EIT images 
	\parencite{adler_impedance_1994}, and may impact the 
	accuracy of a custom model as the patient moves. 
	Correctly modelling the impact of electrode and boundary movement 
	during breathing could help to further improve reconstruction accuracy.

	\item When creating meshes from CT images, errors often occur when electrode placement does not 
	work well on the irregular boundary. 
	Some existing techniques allow electrode 
	placement on arbitrary mesh surfaces \parencite{grychtol_fem_2013}, 
	but these techniques do not 
	natively offer features for advanced mesh refinement control 
	or use with internal structures. Therefore,  
	future work could focus on developing a method to facilitate meshing and electrode
	placement on arbitrary boundary shapes.

	\item Further work is required to determine the safety requirements for internal electrodes in human use.
	The IEC guidelines \parencite{international_electrotechnical_commission_iec_2021} have strict
	requirements for the injection of electrical current near the heart, and the current density on 
	internal electrodes should be examined to ensure that it does not exceed current safety standards. 
	It is also possible to use internal electrodes without injecting currents internally, 
	but this could reduce the sensitivity benefits found when using internal electrodes.
	Further investigation could determine if there are benefits 
	to using internal electrodes only for recording voltage measurements. 

	\item When reconstructing images using internal electrodes with GREIT, 
	internal electrode motion 
	caused artefacts in the image. 
	It is not clear which aspect of the electrode motion introduces extra artefacts in GREIT.
	Work is planned to create an addition to 
	GREIT that will improve reconstruction 
	with internal electrodes and correct for internal probe movement. 

	\item It has been suggested that the contact impedance of internal 
	electrodes should be matched to the
	external electrodes \parencite{nasehi_tehrani_evaluation_2012}, 
	but it is not clear to what degree
	differences in contact impedance and size impact the reconstruction. 
	An investigation into the ideal size, shape and 
	electrical properties of internal electrode 
	could help to inform their use in a clinical setting. 

	\item The novel technique introduced to correct for probe location uses reconstructed images 
	to determine the position of the internal probe. Further investigation is required to determine 
	if the electrode position can be calculated directly from the reconstruction matrix. Previous
	work has shown that 
	the movement direction of individual electrodes can be 
	calculated directly from the reconstruction matrix 
	\parencite{soleimani_imaging_2006}, but this technique does not reconstruct for position. 
	A comparison between techniques to compare the accuracy of each could help to 
	estimate the probe locaiton more accurately.
\end{itemize}

\section{Conclusion}
This thesis presents tools to analyze and control mesh refinement, generate custom meshes 
of arbitrary geometry, and correct for movement artefacts when reconstructing images using 
internal electrodes. This investigation revealed that customized meshes can increase
the accuracy with which ventilation is reconstructed, and that internal electrodes can be used to increase sensitivity 
in the centre of a subject and reconstruct images in an animal model.  
The presented work contributes to increasing internal sensitivity of 
EIT measurements and demonstrates that refined meshes and internal 
electrodes may improve measures of perfusion and help to make EIT a 
viable tool for continuous perfusion monitoring at the bedside.


%Several techniques have been introduced to improve sensitivity to perfusion, but further research is
%required to combine these techniques and validate their ability to image perfusion. 
%In this thesis we have introduced several building blocks for 
%increasing sensitivity, and independantly analyzed their impact
%accuracy in EIT reconstruction. 
%The results presented this thesis can be used to improve internal sensitivity 
%using EIT, and combining methods presented in this chapter we hope


%Having introduced several techniques to improve sensitivity to 
%perfusion in EIT images, future work
%future work could also be done to validate these techniques. 
%also
%entails validation. 
%Exmeriments using saline injections in an animal model with internal electrodes 
%could be conducted to determine if there is a significant difference between 
%cardiosynchronous impedance changes and bolus injection measures. Additionally validation is
%
%
%
%conducted and are being analyzed for their abaility to increase sensitivity to
%bolus related measure
%
%