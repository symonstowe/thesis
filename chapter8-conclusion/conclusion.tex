\chapter{Conclusion}

This thesis investigated the role of mesh refinement and internal electrodes in 
perfusion imaging. 
EIT has very low sensitivity in the center of the thorax where 
may of the perfusion changes of interest occur. 
The source of the cardiosynchronous EIT signal is also 
controversial, and there are many factors that may contribute to 
the measured perfusion signal. 
This thesis presents several contribution to the field of 
perfusion imaging in EIT.
We investigated the limitations of current EIT perfusion imaging
and compared the difference between measures of perfusion using 
a contrast agent and those relying on the cardiosynchronous signal.
A technique to control mesh refinement is developped to enable meshes
with adequate sensitivity accuracy throughout an entire model. This 
technique can also be used to refine around intenal electrodes 
where the sensitivity is much higher in the center of the model than 
with typical external electrode configurations. 
A 3D EIT configuration is presented with 4 internal electrodes that gives high
sensitivity in the center of the model, and accuratey reconstructs a conductive
target in the presence of Gaussian measurement noise.
Finally, we develop a method to calculate movement of 
an internal probe and correct for positioning error in a model. Electrode 
motion artefacts can add artefact to images that overpower the 
component of interest, and add noise outside of the actual 
conductivity changes. This technique uses information in the reconstructed 
image characteristic of probe movement to estimate the actual probe locaiton 
and create a corrected model that reduced the effect of movement on reconstructed 
images. 

\section{Summary of Findings}

\Fref{chap:chapter-3} compared two established methods of perfusion 
imaging and found that perfusion estimation techniques using 
cardiosynchronous signal were comprable to bolus injection methods. 
Despite the different signal origins, both techniques 
resulted in simular perfusion estimates. 
One of the challenges with perfusion imaging with both bolus injection and
frequency filtering is the contribution of blood flow in the heart. 
In 2D with low sensitivity in the center of a subject 
it is challenging to identify and correctly remove the contribution 
of the heart to the perfusion signal. 

In \fref{chap:chapter-4} we explore techniques to
imrove sensitivity accuracy in meshes and ensure mesh density is adequate 
around electrodes. This chapter recomends that 
when mesh density is dissapated constantly from electrodes 
the balance point 
of the nodes in a mesh should be towards the electrodes.
The nodes should be distributed so that the balance point is 
at 80\% of the distance between the electrode and the 
low sensitivity area in the center of a model.
This recomendation can be used to generate 
meshes that are highly accurate when calculating the sensitivity, with less elements
than meshes that are uniformly refined.  





Note - it is possible that measurements on the internal electrode probe 
should be used only for identification of the pulsatile componenet and not
as part of the reconstructed images

Limitations of the lamb model
hear tlocation and whatnoT

\section{Future Work}

safety of internal electrodes with regards to the IEC stuff

Test and validate the 