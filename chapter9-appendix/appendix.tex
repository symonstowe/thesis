\begin{appendices} 

\chapter{GMSH models} \label{app:geo_stuff}
In order to create models with mesh refinement in GMSH matlab was used
to generate geometry (or .geo) files. Examples of the geometry files for 
a basic tank with 4 square external electrodes and the internal probe 
with muptiple densitities are presented. To use these models in EIDORS 
they must be meshed using GMSH and loaded into eidors 
with \verb!gmsh_read_mesh!. Also see \verb!mat_idx_2_electrode! to specify
electrodes in GMSH meshes. To run these meshes with gmsh from the command line 
copy the text into a file and type: \verb!gmsh <filename.geo>! while in the same
directory.

\section{Tank model}
\begin{Verbatim}[fontsize=\footnotesize]
SetFactory("OpenCASCADE"); 
tank_boundary = newv; 
Cylinder(tank_boundary) = {0, 0, 0,  0, 0, 0.125, 0.25, 2*Pi}; 
Box(2) = {-0.025, 0.225,0.10, 0.05, 0.05,0.025}; 
Box(3) = { 0.025,-0.225,0.10,-0.05,-0.05,0.025}; 
Box(4) = { 0.225, 0.025,0.10, 0.05,-0.05,0.025}; 
Box(5) = {-0.225,-0.025,0.10,-0.05, 0.05,0.025}; 
elecs() = BooleanDifference{ Volume{2:5}; Delete;}{Volume{1};}; 
vol() = BooleanFragments{ Volume{tank_boundary};}{Volume{elecs()};}; 
Coherence; 
Mesh.CharacteristicLengthFromPoints = 0; 
Mesh.CharacteristicLengthFromCurvature = 0; 
Mesh.CharacteristicLengthExtendFromBoundary = 0; 
Field[1] = Attractor; 
Field[1].NNodesByEdge = 1000; 
Field[1].FacesList = {4,5,6,7,9}; 
Field[2] = Threshold; 
Field[2].IField = 1; 
Field[2].LcMin = 0.006670; 
Field[2].LcMax = 0.030830; 
Field[2].DistMin = 0.000000; 
Field[2].DistMax = 0.250000; 
Background Field = 2; 
Mesh 1; 
Mesh 2; 
Mesh 3; 
OptimizeMesh "Gmsh"; 
\end{Verbatim}

\section{Internal probe model}
\begin{Verbatim}[fontsize=\footnotesize]
SetFactory("OpenCASCADE"); 
Box(1) = {0, 0, 0,  16.18, 5, 10};
Sphere(2) = {0, 0, 0, 10};
Dilate {{0, 0, 0}, {0.7, 1, 1.5}} { Volume{2}; }
Sphere(3) = {0, 0, 0, 10.5};
Dilate {{0, 0, 0}, {0.7, 1, 1.5}} { Volume{3}; }
Sphere(4) = {0, 0, 0, 16};
Dilate {{0, 0, 0}, {0.51, 1, 1.5}} { Volume{4}; }
Box(5) = {11  , 0, 0, 5.18, 5, 10}; 
Box(6) = {12  , 0, 0, 4.18, 5, 10}; 
Box(7) = {12.4, 0, 0, 3.78, 5, 10}; 
Cylinder(8) = {0, 0, 5,  8, 0, 0, 0.15, 2*Pi};
Sphere(9) = {8, 0, 5, 0.15};
Cylinder(10) = {4.75, 0, 5,  0.25, 0, 0, 0.15, 2*Pi};
Cylinder(11) = {5.75, 0, 5,  0.25, 0, 0, 0.15, 2*Pi};
Cylinder(12) = {6.75, 0, 5,  0.25, 0, 0, 0.15, 2*Pi};
Cylinder(13) = {7.75, 0, 5,  0.25, 0, 0, 0.15, 2*Pi};
probe_temp()=BooleanUnion{Volume{8};Delete;}{Volume{9};}; 
ext() = BooleanDifference{Volume{1};Delete;}{Volume{probe_temp()};};
tip() = BooleanUnion{ Volume{9};Delete;}{ Volume{13}; Delete;};
probe() = BooleanDifference{Volume{probe_temp()};Delete;}{Volume{10:12};};
probe() = BooleanDifference{Volume{probe()};Delete;}{Volume{tip()}; };
tissue() = BooleanIntersection{Volume{ext()};}{Volume{2};Delete;};
bone_ext_a() = BooleanIntersection{Volume{ext()};}{Volume{3};Delete;};
bone_mid_a() = BooleanIntersection{Volume{ext()};}{Volume{4};Delete;};
bone_cent_a()= BooleanDifference{Volume{ext()};Delete;}{Volume{bone_mid_a};};
bone_cent() = BooleanDifference{Volume{bone_cent_a()};Delete;}{Volume{5};};
bone_mid_a() = BooleanDifference{Volume{bone_mid_a()};Delete;}{ Volume{bone_ext_a};};
bone_ext_a() = BooleanDifference{Volume{bone_ext_a()};Delete;}{ Volume{tissue};};
bone_mid_b() = BooleanDifference{Volume{5};Delete;}{Volume{6};};
bone_ext_b() = BooleanDifference{Volume{6};Delete;}{Volume{7};};
elecs() = BooleanFragments{Volume{probe()};Delete;}{Volume{10:12};Delete;};
elecs() = BooleanFragments{Volume{elecs()};Delete;}{Volume{tip()};Delete;};
Coherence; 
Field[1] = Attractor; 
Field[1].NNodesByEdge = 100; 
Field[1].FacesList = {112,113,124,126,130,136,138,143,146,149,152}; 
Field[2] = Threshold; 
Field[2].IField = 1; 
Field[2].LcMin = 0.03; 
Field[2].LcMax = 0.500; 
Field[2].DistMin = 0.050; 
Field[2].DistMax = 2.00; 
Field[3] = Attractor; 
Field[3].NNodesByEdge = 500; 
Field[3].FacesList = {70, 77, 84, 123}; 
Field[4] = Threshold; 
Field[4].IField = 3; 
Field[4].LcMin = 0.4; 
Field[4].LcMax = 1; 
Field[4].DistMin = 0.100; 
Field[4].DistMax = 0.500; 
Field[5] = Attractor; 
Field[5].NNodesByEdge = 100; 
Field[5].FacesList = {123}; 
Field[6] = Threshold; 
Field[6].IField = 3; 
Field[6].LcMin = 0.1; 
Field[6].LcMax = 1; 
Field[6].DistMin = 0.100; 
Field[6].DistMax = 0.500; 
Field[7] = Min;
Field[7].FieldsList = {2, 4, 6};
Background Field = 7;
Mesh 1;
Mesh 2;
Mesh 3;
Save 'advanced_mesh.msh';
\end{Verbatim}

\chapter{Algorithms} \label{app:appendix-algos}
The following sections show the segmentation methods from \fref{chap:chapter-5}
presented as algorithms for a clear presentation of the segmentation steps. 
\section{External boundary}
The following steps were used to segment the external boundary 
from a single slice 
of a CT image using Matlab 2021b  
with the image processing toolbox:
\begin{algorithm}[H]
	\SetAlgoLined
	\KwIn{image}
	\KwOut{external boundary}
		weiner filter\;
		Set the lung intensity to 0\;
		erode image using disk size 20\;
		reconstruct on image from line 4\;
		dilate with disk of size 20\;
		reconstruct on image from line 6\;
		binarize, thresh = 0.5\;
		fill holes\;
		close using disk size = 2\;
		open using disk size = 5\;
		external boundary = largest object\;
	\caption{Segment the external body boundary.}
\end{algorithm}

%\section{Ribcage}
%The below algorithm is a presnetation of the segmentation methods to extract the 
%ribcage from a selected CT slice.
%\begin{algorithm}[H]
%	\SetAlgoLined
%	\KwIn{image}
%	\KwOut{external boundary}
%		weiner filter\;
%		Set the lung intensity to 0\;
%		erode image using disk size 20\;
%		reconstruct on image from line 4\;
%		dilate with disk of size 20\;
%		reconstruct on image from line 6\;
%		binarize, thresh = 0.5\;
%		fill holes\;
%		close using disk size = 2\;
%		open using disk size = 5\;
%		external boundary = largest object\;
%	\caption{Segment the external body boundary.}
%\end{algorithm}
%
%\section{Lungs}
%This algorithm shoes the final steps taken to segment the lungs using the chest cavity 
%segmentation.
%\begin{algorithm}[H]
%	\SetAlgoLined
%	\KwIn{image}
%	\KwOut{chest cavity}
%	\For{CT slice}{
%		Do something
%	}
%		weiner filter\;
%		Set the lung intensity to 0\;
%		erode image using disk size 20\;
%		reconstruct on image from line 4\;
%		dilate with disk of size 20\;
%		reconstruct on image from line 6\;
%		binarize, thresh = 0.5\;
%		fill holes\;
%		close using disk size = 2\;
%		open using disk size = 5\;
%		external boundary = largest object\;
%	\caption{Segment the external body boundary.}
%\end{algorithm}

\end{appendices}