\chapter{Introduction}

Electrical impedance tomography (EIT) is an imaging modality that uses an arrangement of electrodes to simultaneously apply 
stimulation currents and measure 
the resulting electric potentials. Measurements of potential at the electrodes are used in conjunction with 
prior information to reconstruct images 
of the internal conductivity. In biomedical applications, the variance in conductivity between different 
tissue types and fluids within the body enables non-invasive imaging of functional activity using electrodes on the 
body surface.

As electrical current travels through the body it diffuses away from the electrodes,
travelling in three dimensions along the path of least resistance. 
EIT is most sensitive to impedance changes close to the electrodes 
and along the path of the injected current.
%When reconstructing a single tomographic slice,   
%off-plane impedance changes from a three-dimensional volume project 
%onto a two-dimensional plane yielding more than one solution.
This document proposes a thesis that investigates the use of novel 3D electrode 
configurations and injection patterns to maximize sensitivity and improve continuous monitoring 
of cardiopulmonary measures using EIT.

\section{Problem}

Electrical Impedance Tomography (EIT) is sensitive to perfusion 
changes in tissue and the flow of blood in the thorax.
The source of the impedance change is not well understood. 
How does the magnitude of the impedance change due to perfusion
compare to the impedance change due to motion in the thorax at the cardiac frequency? 





\section{Thesis objectives}
The following questions will be answered by the thesis:
\begin{enumerate}
	\item How do current methods of EIT lung perfusion imaging compare? (\emph{Chapter 2})
	\item What advantages can internal electrode configurations provide? (\emph{Chapter 3})
\item What advantage do alternative external electrode configurations provide
	when monitoring aortic blood flow? (\emph{Chapter 4})
\item How can a configuration that is better able to isolate cardiac activity
	improve measures of lung perfusion and aortic flow in an animal model? (\emph{Chapter 5})
\end{enumerate}

\section{Contributions}


\section{Publications}
Published works related to the thesis:
\begin{itemize}
	\item 
\end{itemize}

Other projects:
\begin{itemize}
	\item \fullcite{stowe_monitoring_2019}.
\end{itemize}
