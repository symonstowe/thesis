\chapter{Introduction}

Electrical impedance tomography (\acrshort{eit}) is an imaging modality that uses an arrangement 
of electrodes to simultaneously apply 
stimulation currents and measure 
the resulting electric potentials. Measurements of potential at the 
electrodes are used in conjunction with 
prior information to reconstruct images 
of the internal conductivity. In biomedical applications, 
the variance in conductivity between different 
tissue types and fluids enables non-invasive 
imaging of functional activity using electrodes on the 
body surface. 

Thoracic EIT focuses on imaging cardiopulmonary activity
using a belt of electrodes placed around the ribcage.
In images of the chest, the signal is often dominated by 
impedance changes due to air movement in the lungs. 
The much smaller signals due to blood flow are of great interest for 
continuous monitoring and diagnostic applications,
but are challenging to identify. 

This thesis explores techniques to improve sensitivity to blood movement, or perfusion,  
in EIT through the development of novel meshing techniques and electrode configurations
incorporating internal electrodes. 
We investigate the current limitations of perfusion monitoring in 2D, and
analyze novel meshing techniques and electrode placements for their 
ability to improve thoracic imaging and increase sensitivity to blood flow. 

\section{Motivation}
\TODO{Briefly \\
Why should we care?\\
Why to we measure perfusion? \\
What are the benefits if this technology works well? (increased sensitivity?) \\
What are the benefits of using EIT for this?}

\section{Problem}

EIT can be used to image blood movement in two ways. 
First, it is possible to track the flow of blood using a 
conductivity-contrasting agent injected into a vein or artery
and second, filtering the signal to isolate cardiac-frequency 
impedance changes. 
Using a conductive bolus injection to image perfusion has been well established and 
can give an easily detectable signal when using large volumes
or high concentrations of the contrasting agent, but they do not 
allow continuous monitoring and frequent injections may pose 
risks to the patient.
In thoracic EIT bolus injections also typically occur during apnea to 
facilitate removal from the respiratory component.
Filtering techniques for perfusion imaging are much more appealing for 
monitoring applications as they are less invasive 
and could be used continuously, but they present several challenges. 

The primary challenge is the small amplitude of the signal. 
Impedance changes related to cardiac activity are often an order of
magnitude smaller than signals related to respiration and can be challenging to identify without additional signals like 
an \acrfull{ecg}, and averaging many heartbeats together. When 
averaging over several heartbeats, the ability of the system to 
monitor in real time is greatly diminished. Additionally 
there is some uncertainty surrounding the source of
cardiac-frequency impedance changes. It is unclear to what 
extent the impedance changes stem from pulsatile
motion in the thorax compared to the movement of blood, 
and if they can be used as a true measure of perfusion. 

\acrshort{eit} is minimally invasive, requiring only 
the application of electrodes to the body surface, but 
does not have high sensitivity in the central regions 
of the chest where perfusion changes are likely to occur.
As electrical current travels through the body it diffuses away from 
the electrodes,
travelling in three dimensions along the paths of least resistance. 
EIT is most sensitive to impedance changes close to the electrodes
where the current density is highest, 
and along the path of the injected current.

The placement of electrodes internally has the 
potential to increase current density and sensitivity in 
the center of the thorax. 
The benefits of internal electrodes have been simulated 
in 2D showing great improvemets in reconstruction accuracy 
and sensitivity~\parencite{tehrani_modelling_2012}, but 
in practice there are several challenges to overcome. 
Due to differences in physiology it is challenging to model
the correct location of the internal electrodes between subjects
and electrodes placed internally may move with relation 
to the external electrodes during different physiological
processes.  
To accurately reconstruct images with internal electrodes 
an accurate model is required with precise placement of electrodes
and a model that matches the subject and is refined 
to meet the accuracy requirements. 
It is in part due to these challenges that internal electrodes 
have not been widely used in real-world situations, and to our knowledge 
no implementation of internal electrodes in conjunction 
with a 3D arrangement of external electrodes has been used for \emph{in-vivo}
imaging.

The applications of an EIT system with increased sensitivity to perfusion changes
in the thorax are extensive. Methods to monitor blood pressure non-invasively are 
currently under development and could be greatly improved with increased sensitivity 
near the region of interest. Increased sensitivity near the heart 
could enable EIT to be realized as a low-cost solution to monitor and image
hemodynamic activitiy more accurately.  

\section{Thesis objectives}
The goal of this project is to improve sensitivity of EIT to perfusion using 
improved mesh accuracy and custom electrode locations. This thesis approaches problem 
from two angles (\fref{fig:goal}). We investigate both
advanced meshing techniques, and novel electrode configurations
with internal electrodes and their applications to improve thoracic EIT.

\begin{figure}
	\centering
\begin{tikzpicture}
<code goes here>
\end{tikzpicture}\begin{tikzpicture}[
  level 1/.style={sibling distance=50mm},
  edge from parent/.style={->,draw},
  >=latex]

% root of the the initial tree, level 1
\node[root] {Improve EIT measures of perfusion}
% The first level, as children of the initial tree
  child {node[level 2] (c1) {1) Evaluation of current perfusion monitoring}}
  child {node[level 2] (c2) {2) Advanced meshing techniques}}
  child {node[level 2] (c3) {3) Novel electrode positioning}};

% The second level, relatively positioned nodes
\begin{scope}[every node/.style={level 3}]
\node [below of = c1, xshift=12pt, yshift=-27pt] (c11) {Bolus- vs 
frequency-based measures of perfusion \emph{(\Fref{chap:chapter-3})}}; 

\node [below of = c2, xshift=12pt, yshift=-19pt] (c21) 
{Controlling mesh refinement\\ \emph{(\Fref{chap:chapter-4})}};
\node [below of = c21, yshift=-25pt] (c22) 
{Creating Custom meshes from CT images \emph{(\Fref{chap:chapter-5})}};
\node [below of = c3, xshift=12pt, yshift=-12pt] (c31) 
{Internal electrodes in 3D \emph{(\Fref{chap:chapter-6})}};
\end{scope}

% lines from each level 1 node to every one of its "children"
\foreach \value in {1}
  \draw[->] (c1.195) |- (c1\value.west);

\foreach \value in {1,2}
  \draw[->] (c2.195) |- (c2\value.west);

\foreach \value in {1}
  \draw[->] (c3.195) |- (c3\value.west);

\end{tikzpicture}
\caption[Overfiew of thesis objectives]{We aim to improve EIT measures of perfusion through 3 avenues.
1) Comparing and investigating the limitaitons of existing perfusion imaging methods;
2) Advancing meshing techniques; and 
3) Using novel electrode locations consisting of a 3D external
configuration with internal electrodes.}
\label{fig:goal}
\end{figure}

%\section{Contributions}
%The following 
%
%This thesis presents several contributions to the field of perfusion imaging and 
%meshing in EIT.
%
%\COMMENT{OLD stuff}
%The following questions will be answered by the thesis:
%\begin{enumerate}
%	\item How do current methods of EIT lung perfusion imaging compare? (\emph{Chapter 2})
%	\item What advantages can internal electrode configurations provide? (\emph{Chapter 3})
%	\item What advantage do alternative external electrode configurations provide
%	when monitoring aortic blood flow? (\emph{Chapter 4})
%	\item How can a configuration that is better able to isolate cardiac activity
%	improve measures of lung perfusion and aortic flow in an animal model? (\emph{Chapter 5})
%\end{enumerate}
%
%
%\section{Publications}
%Published works related to the thesis:
%\begin{itemize}
%	\item \fullcite{stowe_comparison_2019}
%	\item \fullcite{stowe_effect_2020}
%	\item \fullcite{stowe_generating_2021} 
%\end{itemize}
%
%Other projects:
%\begin{itemize}
%	\item \fullcite{stowe_monitoring_2019}.
%\end{itemize}
%